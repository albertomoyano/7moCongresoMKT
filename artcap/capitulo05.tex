\chapter{Propiedades Fundamentales de los Números Reales}

\section{Introducción}

En este capítulo, exploraremos algunas de las propiedades básicas de los números reales. Los números reales forman la base de gran parte del análisis matemático, y su estructura está regida por axiomas y teoremas fundamentales que permiten deducir propiedades importantes.

Comenzaremos con algunas definiciones y axiomas, para luego pasar a teoremas que establecen relaciones clave entre estos números.

\section{Teoremas y Lemas}

El siguiente teorema establece una propiedad simple, pero esencial, de los números reales cuando son iguales.

\begin{theorem}
Sea $a$ y $b$ dos números reales. Si $a = b$, entonces $a^2 = b^2$.
\end{theorem}

\begin{proof}
Dado que $a = b$, al elevar ambos lados de la igualdad al cuadrado obtenemos:

\[
a^2 = b^2.
\]

Esto sigue directamente de la definición de la igualdad y la operación de elevar al cuadrado un número real.
\end{proof}

Sin embargo, la implicación inversa requiere un análisis más cuidadoso, lo que nos lleva al siguiente lema.

\begin{lemma}
Si $a^2 = b^2$, entonces $a = b$ o $a = -b$.
\end{lemma}

\begin{proof}
Si $a^2 = b^2$, entonces podemos reescribir la ecuación como:

\[
a^2 - b^2 = 0.
\]

Esto se factoriza como:

\[
(a - b)(a + b) = 0.
\]

Por lo tanto, para que el producto sea cero, debe cumplirse que $a - b = 0$ (es decir, $a = b$) o que $a + b = 0$ (es decir, $a = -b$).
\end{proof}

\section{Proposiciones y Corolarios}

Ahora que hemos establecido las relaciones básicas entre números reales y sus cuadrados, podemos generalizar este resultado a potencias mayores con el siguiente resultado.

\begin{proposition}
Si $a = b$, entonces $a^n = b^n$ para cualquier $n \in \mathbb{N}$.
\end{proposition}

\begin{proof}
Esto sigue directamente de la propiedad de la igualdad. Si $a = b$, entonces al elevar ambos lados de la ecuación a una potencia $n$ obtenemos:

\[
a^n = b^n.
\]

Esto es válido para cualquier $n$ en el conjunto de los números naturales.
\end{proof}

Un caso particular interesante de esta proposición es cuando $a = 0$, lo cual nos lleva al siguiente corolario.

\begin{corollary}
Si $a = 0$, entonces $a^n = 0$ para cualquier $n \in \mathbb{N}$.
\end{corollary}

\section{Definiciones y Ejemplos}

Antes de continuar, es útil recordar la definición del conjunto de los números reales.

\begin{definition}
El conjunto $\mathbb{R}$ se define como el conjunto de todos los números reales. Este conjunto incluye tanto los números racionales como los irracionales, y forma la base de muchas ramas de las matemáticas.
\end{definition}

A continuación, presentamos un ejemplo concreto de los resultados anteriores.

\begin{example}
Consideremos el número $a = 3$ y $b = 3$. Según el Teorema 1, como $a = b$, entonces $a^2 = b^2 = 9$. Este es un ejemplo directo de cómo se aplican los resultados a casos específicos.
\end{example}

\section{Axiomas y Conjeturas}

Finalmente, presentamos un axioma básico de los números reales, que se utiliza frecuentemente en diversas demostraciones.

\begin{axiom}
Para cualquier número real $x$, se cumple que $x + 0 = x$.
\end{axiom}

Este axioma refleja la propiedad de identidad aditiva, que es una de las propiedades fundamentales de los números reales.

También se incluye una conjetura famosa que ha sido objeto de estudio durante siglos en matemáticas.

\begin{conjecture}
No existen números reales $x$ e $y$ tales que $x^3 + y^3 = z^3$ para algún $z \in \mathbb{R}$.
\end{conjecture}

Esta conjetura es conocida como una versión del Último Teorema de Fermat, una conjetura que fue resuelta por Andrew Wiles en 1994 para el caso de los números enteros, pero su validez para números reales sigue siendo un tema abierto en ciertos contextos.

\section{Observaciones Finales}

\begin{remark}
Este conjunto de resultados es fundamental en el análisis de ecuaciones y propiedades en el cálculo diferencial. Las propiedades de los números reales, junto con los axiomas básicos que los rigen, proporcionan una base sólida para entender fenómenos más complejos en matemáticas.
\end{remark}

