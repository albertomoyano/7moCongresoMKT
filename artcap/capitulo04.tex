\chapter{El ADN: estructura química y función}

\section{Introducción}

El ácido desoxirribonucleico (ADN) es una molécula biológica fundamental que almacena la información genética de los organismos vivos. Desde una perspectiva química, el ADN es un polímero formado por subunidades conocidas como nucleótidos. Cada nucleótido está compuesto por tres elementos principales: un grupo fosfato, una base nitrogenada y un azúcar de cinco carbonos llamado desoxirribosa. El estudio del ADN combina aspectos tanto de la química orgánica como de la bioquímica para explicar su estructura y función.

\section{Estructura química del ADN}

El ADN está compuesto por una doble hélice formada por dos cadenas de nucleótidos que se enrollan entre sí. Cada cadena está constituida por un esqueleto de grupos fosfato y desoxirribosa, mientras que las bases nitrogenadas se proyectan hacia el interior de la hélice. Las bases nitrogenadas se dividen en dos tipos: purinas (adenina \textbf{(A)} y guanina \textbf{(G)}) y pirimidinas (citosina \textbf{(C)} y timina \textbf{(T)}).

Los nucleótidos están unidos por enlaces fosfodiéster entre el grupo fosfato de un nucleótido y el carbono 3' del azúcar del siguiente nucleótido. Este enlace se representa químicamente como:

\begin{equation*}
\chemfig{R-[:0]P(=[:90]O)(-[:180]O^-)-[:0]O-[:0]R'}
\end{equation*}

Donde $R$ y $R'$ representan los fragmentos de nucleótidos adyacentes. La doble hélice se mantiene unida por enlaces de hidrógeno entre pares específicos de bases. La adenina siempre forma dos enlaces de hidrógeno con la timina, mientras que la guanina forma tres enlaces con la citosina, según la regla de complementariedad de las bases. Esto se puede escribir como:

\begin{equation*}
\text{A-T:} \quad \text{2 enlaces de H}, \quad \text{G-C:} \quad \text{3 enlaces de H}
\end{equation*}

\section{Estructura tridimensional del ADN}

El ADN tiene una forma helicoidal que fue descrita por primera vez en 1953 por James Watson y Francis Crick. Esta estructura es conocida como la doble hélice. En la forma B del ADN, que es la más común en condiciones fisiológicas, las dos hebras se enrollan con un giro hacia la derecha. Las bases nitrogenadas se apilan en el interior de la hélice, aproximadamente a una distancia de 0.34 nanómetros entre sí, mientras que la hélice completa tiene un paso de 3.4 nanómetros, es decir, cada diez pares de bases forman un giro completo.

\begin{figure}[h!]
\centering
\includegraphics[scale=0.4]{double_helix.png}
\caption{Estructura de la doble hélice del ADN, mostrando el esqueleto de fosfato-desoxirribosa y el apareamiento de bases nitrogenadas.}
\end{figure}

\section{Las bases nitrogenadas: Fórmulas y estructuras}

Las bases nitrogenadas son derivadas de dos estructuras fundamentales: la purina y la pirimidina. La fórmula general de una purina es la siguiente:

\begin{equation*}
\chemfig{*6(=-N(-H)-=N-=-N-=)}
\end{equation*}

Mientras que la pirimidina se representa de la siguiente manera:

\begin{equation*}
\chemfig{*6(=-N(-H)-=N-=-)}
\end{equation*}

Estas bases se unen entre sí por enlaces de hidrógeno según los siguientes esquemas:

\begin{figure}[h!]
\centering
\chemfig{N*6(-N=-N=-N)}
\quad
\chemfig{N*6(=-N=-N=-N)}
\caption{Estructura química de la adenina (izquierda) y la timina (derecha).}
\end{figure}

\section{Replicación del ADN}

Una de las propiedades más importantes del ADN es su capacidad para replicarse de manera precisa. Este proceso es fundamental para la herencia genética y ocurre antes de la división celular. Durante la replicación, las dos cadenas de la hélice se separan y sirven como plantillas para la síntesis de nuevas cadenas complementarias. La enzima ADN polimerasa cataliza la adición de nuevos nucleótidos siguiendo las reglas de complementariedad de bases.

El proceso de replicación puede resumirse de la siguiente manera:

\begin{enumerate}
\item Las hebras del ADN parental se separan.
\item Las nuevas cadenas se sintetizan añadiendo nucleótidos complementarios.
\item El resultado es la formación de dos moléculas de ADN idénticas, cada una con una hebra original y una nueva.
\end{enumerate}

\section{Conclusión}

Desde una perspectiva química, el ADN es una molécula compleja que alberga la información necesaria para el desarrollo y funcionamiento de los organismos vivos. Su estructura, basada en la complementariedad de las bases nitrogenadas y los enlaces fosfodiéster, permite la transmisión precisa de información genética durante la replicación. A medida que se han desarrollado nuevas técnicas de análisis molecular, se ha profundizado en el estudio del ADN, lo que ha abierto nuevas posibilidades para el avance de la biotecnología y la medicina.

